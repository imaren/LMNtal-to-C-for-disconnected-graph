\section{例題の定義}

\subsection{LMNtal プログラム}

プログラムには,初期状態として,1つのハイパーリンクを引数として持つアトム \texttt{start\_1} 1つと,
整数,ハイパーリンク,ハイパーリンク,整数の組を引数として持つアトム \texttt{distance\_4} がいくつか与えられる.

\lstinputlisting{progs/time-point.lmn}

\subsection{各アトムとプログラムの数学としての意味}

プログラム内には,\texttt{between\_3}, \texttt{distance\_4} が,主要なアトムとして出現し,引数に含まれるハイパーリンクは,整数上の変数を意味する.
\texttt{between\_3} の引数は,整数 $a$ ,ハイパーリンク $H$,整数 $b$の組であり,$a \leq H \leq b$ を意味する.
\texttt{distance\_4} の引数は,整数 $a$, ハイパーリンク $H_1$,ハイパーリンク $H_2$,整数 $b$ の組であり, $a \leq H_2 - H_1 \leq b$ を意味する.

$min \leq H_1 \leq max$ かつ $a \leq H_2 - H_1 \leq b$ であるとき,
$a+min \leq H_2 \leq b+max $ がわかる.
また,$min \leq H_2 \leq max$ かつ $a \leq H_2 - H_1 \leq b$ であるとき,
$max-b \leq H_1 \leq min-a $ がわかる.


ある変数の集合 $V$ を考える.
今,
$0 \leq H_0 \leq 0$ where $H_0 \in V$ を表す \texttt{between\_3} アトムが1つと,
$a \leq H_j - H_i \leq b$ where $H_i \in V, H_j \in V, a \in \mathbb{Z}, b \in \mathbb{Z}$ を表す \texttt{distance4} アトムが複数与えられるとする.

このとき,計算可能な全ての変数 $H \in V$ について,$min \leq H \leq max$ を計算する.

\subsection{このプログラムの実用例題的な意味(問題文)}
あなたは部署の依存関係のあるイベントの時間管理を任されました.
あなたには,複数あるイベントの内 2つのイベントについて,
$\text{Event}_j$ が必ず $\Event{Task}_i$ の $a$ 分後から $b$ 分後までの間に実行されるという関係が
複数知らされます.
また,どのイベントが始めに行われるべきかということが知らされます.
始めに行われるべきイベントの時間をイベントの開始後 0 分後 から 0 分後であるとすると,
各イベントは開始後何分後から何分後に実行されるでしょうか.

%% タスク -> イベント に変更 イベントを起こす時間に制約条件が付いている
%% 各イベントを発生させる 最小と最大の時間を求める
%% forward をやってみて,backward をやってみる.イメージは一回前向きにやってみたらもう一回逆向きにやってみる
%% 制約プログラミングの問題,イベントの問題なら転がってるので作ってみても OK
%% 
%% backward を使って forward